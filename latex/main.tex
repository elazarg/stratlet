\documentclass[11pt]{article}

\usepackage[margin=1in]{geometry}
\usepackage{amsmath,amssymb,amsthm,mathtools,stmaryrd}
\usepackage{enumitem}
\usepackage{mathrsfs}

\newcommand{\ctx}{\Gamma}
\newcommand{\Env}{\mathsf{Env}}
\newcommand{\Val}{\mathsf{Val}}
\newcommand{\Ty}{\mathsf{Ty}}
\newcommand{\Prog}{\mathsf{Prog}}
\newcommand{\WDist}{\mathsf{WDist}}
\newcommand{\supp}{\mathsf{supp}}
\newcommand{\mass}{\mathsf{mass}}
\newcommand{\EU}{\mathsf{EU}}
\newcommand{\Reach}{\mathsf{Reach}}
\newcommand{\WF}{\mathsf{WF}}
\newcommand{\View}{\mathsf{View}}
\newcommand{\proj}{\mathsf{proj}}
\newcommand{\Act}{\mathsf{Act}}
\newcommand{\YieldId}{\mathsf{YieldId}}
\newcommand{\Player}{\mathsf{Player}}
\newcommand{\Profile}{\mathsf{Profile}}

\title{Vegas / StratLet: Core Definitions (ProtoLet Milestone)}
\author{}
\date{}

\begin{document}
\maketitle

\section{Ambient setup (typed environments)}
Assume a typed context calculus with:
\begin{itemize}[leftmargin=*]
\item a set of types $\Ty$ and, for each $\tau\in\Ty$, a set of values $\Val(\tau)$,
\item contexts $\ctx$ (finite lists of types),
\item environments $\Env(\ctx)$ (tuples of values matching $\ctx$).
\end{itemize}
We treat context extension implicitly: an environment in an extended context carries the previous environment as a prefix in the standard way.

\section{Finite-support weighted distributions}
A \emph{finite-support weighted distribution} over a set $A$ is a finite multiset (or list) of weighted outcomes:
\[
  d \in \WDist(A) \;\; \equiv \;\; \{(a_i,w_i)\}_{i=1}^n,\quad a_i\in A,\; w_i\in \mathbb{R}_{\ge 0}.
\]
Representations may contain ``junk'' outcomes with weight $0$.

\paragraph{Support-within predicate.}
For a predicate $S : A\to \mathsf{Prop}$, define:
\[
  \mathsf{SupportedOn}(d,S)
  \;\;:\Leftrightarrow\;\;
  \forall (a,w)\in d,\; (w\neq 0 \Rightarrow S(a)).
\]
Equivalently, ignoring the representational junk, $\supp(d)\subseteq \{a \mid S(a)\}$.

\paragraph{Monadic structure (informal).}
We assume the standard finite-support weighted monad operations:
\begin{itemize}[leftmargin=*]
\item $\mathsf{pure}(a)$: singleton $\{(a,1)\}$,
\item $\mathsf{bind}(d,f)$: for each $(a,w)$ in $d$ and each $(b,u)$ in $f(a)$, output $(b,wu)$,
\item $\mathsf{scale}(c,d)$: multiply all weights by $c$,
\item $\mathsf{zero}$: empty distribution.
\end{itemize}
We also use $\mathsf{observe}(\varphi,d)$ as hard rejection (filtering outcomes satisfying $\varphi$), yielding an \emph{unnormalized} subdistribution. When needed, a separate $\mathsf{normalize}$ operator can convert a subdistribution into a conditional distribution (when total mass is nonzero).

\section{A core ``let-calculus with effects''}
We consider a typed program calculus $\Prog$ whose details are conventional:
\begin{itemize}[leftmargin=*]
\item deterministic let-binding,
\item effectful bind against a command,
\item statements (including observation / failure),
\item return.
\end{itemize}
Semantics is given by an evaluator $\llbracket p\rrbracket$ parameterized by a \emph{handler} for commands, yielding a result in some effect carrier. The probabilistic instantiation uses $\WDist$ as the carrier.

(Concrete de Bruijn details, weakening/renaming lemmas, etc.\ are standard and omitted here.)

\section{ProbLet: probabilistic commands}
ProbLet specializes the effect commands to a probabilistic interface.

\subsection{Kernels}
A \emph{kernel} in context $\ctx$ producing a value of type $\tau$ is a function
\[
  K : \Env(\ctx)\to \WDist(\Val(\tau)).
\]
Intuitively: given the current environment, produce a finite-support distribution of outcomes.

\subsection{Sampling and observation}
The command $\mathsf{sample}(K)$ draws from $K(\rho)$ at runtime environment $\rho$.
Observation is modeled as hard rejection: if the predicate fails, the execution path receives weight $0$ (equivalently, the path is removed from support).

Thus ProbLet programs denote \emph{subdistributions} in general, unless a normalization step is applied externally.

\section{ProtoLet: protocol-with-yields core}
ProtoLet extends the core calculus by making interaction points explicit and stable:
\begin{itemize}[leftmargin=*]
\item each interaction site carries a stable identifier $\YieldId$,
\item each decision site is annotated with an explicit \emph{view} limiting information dependence,
\item chance and decision yields are separated in the syntax.
\end{itemize}

\subsection{Views}
A \emph{view} at ambient context $\ctx$ is a projection to a (typically smaller) context $\Delta$:
\[
  v \in \View(\ctx)
  \quad\text{carries}\quad
  \Delta = v.\Delta,\;\;
  v.\proj : \Env(\ctx)\to \Env(\Delta).
\]
This is the fundamental partial-information mechanism: strategies may depend only on $v.\proj(\rho)$, not on the full environment $\rho$.

\subsection{Actions}
At a decision yield with view $v$ and result type $\tau$, an \emph{action set} is a map
\[
  A \in \Act(v,\tau)
  \quad\text{meaning}\quad
  A : \Env(v.\Delta)\to \mathcal{P}_\mathrm{fin}(\Val(\tau)),
\]
assigning to each observation (view-environment) a finite set of legal actions.

\subsection{Commands: chance vs.\ decision yields}
ProtoLet commands (schematically) include:
\begin{itemize}[leftmargin=*]
\item \textbf{Chance yield}:
  \[
    \mathsf{sample}(id, v, K)
    \quad\text{where}\quad
    K:\Env(v.\Delta)\to \WDist(\Val(\tau)).
  \]
  (In many uses $v$ is the identity view; $v$ is carried uniformly for compositionality.)
\item \textbf{Decision yield}:
  \[
    \mathsf{choose}(id, who, v, A)
  \]
  where $who\in \Player$ identifies the decision-maker, $v$ is the view, and $A$ is the legal action specification.
\end{itemize}

\subsection{Profiles (strategies)}
A \emph{profile} supplies, for each decision yield instance, a kernel over the \emph{view environment}:
\[
  \sigma \in \Profile
  \quad\text{provides}\quad
  \sigma.\mathsf{choose}(who,id,v,A): \Env(v.\Delta)\to \WDist(\Val(\tau)).
\]
This is intentionally \emph{not} restricted to be supported on $A$ by construction; legality is imposed separately via well-formedness.

\paragraph{Semantics.}
Given a profile $\sigma$, ProtoLet programs evaluate to $\WDist$ by interpreting:
\begin{itemize}[leftmargin=*]
\item $\mathsf{sample}(id,v,K)$ as sampling from $K(\text{obs})$ with $\text{obs}=v.\proj(\rho)$,
\item $\mathsf{choose}(id,who,v,A)$ as sampling from $\sigma.\mathsf{choose}(who,id,v,A)(\text{obs})$ with $\text{obs}=v.\proj(\rho)$.
\end{itemize}

\subsection{Supported-on legality at a yield}
For a distribution $d$ and predicate $S$, recall $\mathsf{SupportedOn}(d,S)$ ignores zero-weight outcomes.

\paragraph{Legality at a specific decision yield instance.}
For $\sigma$, $(who,id,v,A)$, and an observation $\mathsf{obs}\in \Env(v.\Delta)$:
\[
  \mathsf{LegalAt}(\sigma,who,id,v,A,\mathsf{obs})
  \;:\Leftrightarrow\;
  \mathsf{SupportedOn}\big(\sigma.\mathsf{choose}(who,id,v,A)(\mathsf{obs}),
    \; a\mapsto (a\in A(\mathsf{obs}))\big).
\]
Thus legality means: all nonzero-weight actions the strategy might select are legal.

\subsection{Reachability specifications}
A \emph{reachability specification} is a family of predicates on environments:
\[
  \Reach \;:\; \forall \ctx,\; \Env(\ctx)\to \mathsf{Prop}.
\]
Milestone default is $\mathsf{ReachAll}(\rho)\equiv \mathsf{True}$ (everything reachable). More refined instantiations can encode semantic reachability later.

\subsection{Program-relative well-formedness}
Well-formedness checks legality only at \emph{choose sites that syntactically occur in the program}.

Define $\WF(\Reach,\sigma,p)$ recursively on program structure:
\begin{itemize}[leftmargin=*]
\item $\WF(\Reach,\sigma,\mathsf{return}(\cdot)) \equiv \mathsf{True}$,
\item structural cases ($\mathsf{letDet}$, statements) recurse into the continuation,
\item for effect bind $\mathsf{doBind}(c,k)$:
\begin{itemize}[leftmargin=*]
\item if $c=\mathsf{sample}(\cdots)$ then no additional obligation; recurse into $k$,
\item if $c=\mathsf{choose}(id,who,v,A)$ in ambient context $\ctx$, require
\[
  \forall \rho\in \Env(\ctx),\;
    \Reach(\rho)\Rightarrow
      \mathsf{LegalAt}(\sigma,who,id,v,A, v.\proj(\rho)),
\]
and also recurse into $k$.
\end{itemize}
\end{itemize}

\paragraph{Interpretation.}
$\WF(\Reach,\sigma,p)$ means: at every decision yield appearing in $p$, the profile $\sigma$ only plays legal actions on environments deemed reachable by $\Reach$.

\section{From ProtoLet to ProbLet (strategy application)}
ProtoLet can be compiled to ProbLet by \emph{discharging decision yields} into probabilistic sampling using a supplied (possibly partial) profile.

\subsection{Partial profiles and application}
A \emph{partial profile} $\pi$ assigns a kernel to some (but not necessarily all) decision yield instances:
\[
  \pi.\mathsf{choose?}(who,id,v,A) \in \mathsf{Option}(\Env(v.\Delta)\to \WDist(\Val(\tau))).
\]
Applying $\pi$ to a program replaces any decision yield covered by $\pi$ with a chance yield that samples from the provided kernel, leaving uncovered decision yields intact. (This is the ``strategy application'' pass.)

\paragraph{Extends relation (informal).}
A total profile $\sigma$ \emph{extends} a partial profile $\pi$ if, wherever $\pi$ provides a kernel, $\sigma$ uses the same kernel.

A key sanity property (the intended lemma) is:
\[
  \sigma \text{ extends } \pi
  \;\Rightarrow\;
  \llbracket p\rrbracket_\sigma
  =
  \llbracket \mathsf{applyProfile}(\pi,p)\rrbracket_\sigma.
\]
This is the formal statement that partial strategy application does not change behavior for completions that agree with the partial profile.

\subsection{Compilation to ProbLet when no decisions remain}
For programs with no remaining decision yields (syntactic predicate $\mathsf{NoChoose}(p)$), compilation to ProbLet is defined by mapping all remaining yields to ProbLet sampling, producing a pure probabilistic program whose semantics is the same $\WDist$ denotation.

\section{Utilities, expected utility, and Nash (discrete)}
ProtoLet can be turned into a game by fixing utilities on terminal outcomes.

\subsection{Utility and expected utility}
Let terminal value type be $\tau_\mathrm{out}$ and suppose a utility function for a player $who$:
\[
  u_{who} : \Val(\tau_\mathrm{out}) \to \mathbb{R}.
\]
Given a subdistribution $d\in \WDist(\Val(\tau_\mathrm{out}))$, define \emph{raw} expected utility:
\[
  \EU_{\mathrm{raw}}(d,u_{who}) \;\;=\;\; \sum_{(a,w)\in d} w\cdot u_{who}(a).
\]
If conditioning is desired, one can define $\EU_{\mathrm{cond}}$ by normalizing $d$ first (when $\mass(d)\neq 0$), but the raw form is the basic discrete definition.

\subsection{Deviations and well-formed Nash}
Given a baseline profile $\sigma$, a deviation for player $who$ is another profile $\sigma'$ that differs from $\sigma$ only on that player’s decision yields. (Exact representation is conventional.)

A \emph{WF Nash equilibrium} (relative to reachability $\Reach$) is:
\[
  \mathsf{IsNash}_{\WF}(\Reach,\sigma,p,u)
  \;:\Leftrightarrow\;
  \WF(\Reach,\sigma,p) \;\wedge\;
  \forall who,\;\forall \sigma' \text{ deviating for } who,\;
    \WF(\Reach,\sigma',p)\Rightarrow
      \EU\big(\llbracket p\rrbracket_{\sigma}, u_{who}\big)
      \ge
      \EU\big(\llbracket p\rrbracket_{\sigma'}, u_{who}\big).
\]
This encodes the usual Nash condition but restricts attention to \emph{well-formed} deviations (legal play on reachable environments).

\section*{What this milestone commits to (and what it does not)}
This layer commits to:
\begin{itemize}[leftmargin=*]
\item explicit yields with stable IDs and explicit views,
\item semantics into finite-support weighted subdistributions,
\item a strategy/profile interface and a legality/WF discipline,
\item compilation to a probabilistic core when decisions are discharged.
\end{itemize}
It does \emph{not} yet commit to:
perfect recall enforcement, semantic reachability, measure-theoretic EU, cryptographic commitment realizations, or causal parent-set specifications (these can be layered above).

\end{document}
